\section{Discussion}
Taking the quality of a relationship into consideration lead to an improved model to explain death and the emergence of a king in \got. While the number of connections explain a big part, the support or aversion network is worth to be considered as well. The overall network seems to explain the development of the show's characters fairly well and also allows for an outlook to who might become king or queen in the new season.\\
The result of such a network however is always quite subjective, beginning with the reliability (and thus validity) due to my own rating of the social ties. Another very subjective step was the arrangement of nodes in \texttt{igraph} with \texttt{tkplot()}, the interactive graphing feature of the package. Knowing the characters made the arrangement much easier. The interpretation of this network might have then lead to the classy hindsight bias \citep{TVERSKY1973207}, in which one easily explains previous events because one knows the end of the story. The knowledge of who died definitely facilitated the analysis of how and why connections might have lead to a character's death. The estimate of a future king or queen is the only scientific statement that can be proven false in the future season 8.

\comm{Main content:
- Are the estimates by the authors realistic?
- reliability: would we be able to replicate the model with newer data? Would the construction of the model still work or is it a one-time-only model that only works with the data from 2009?
- what to enhance in the paper?
- convergence of the model? what do the plots say?}


\begin{figure}
    \centering
    \includegraphics[scale=0.2]{tyrion.jpeg}
    \caption{Tyrion Lannister, one of the most central nodes according to a number of centrality measures, is often quoted when referring to his enormous knowledge and sociability. He is a good example for how connections and knowledge lead to power and influence.}
    \label{fig:tyrion}
\end{figure}
