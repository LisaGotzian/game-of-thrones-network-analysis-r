\section{Results} \label{sec:results}
\subsection{Overview}
Table \ref{table:data} serves as an adjacency list to create the social network of the first season of the show. For figure \ref{fig:GOTplot}, each line in the table corresponds to an undirected edge or link while each source or target becomes a node or actor. The weight is used to indicate the strength of the edge drawn, the "+" or "-" determine the color used in the graph. Additionally, all dead characters are greyed out.
The resulting plot creates a concise picture of how the positive and negative ties interplay with the actors’ power and how he or she is perceived within the network.
The full code can be found in the appendix, the code in listing \ref{lstlisting:plot} creates figure \ref{fig:GOTplot}.\\

\begin{lstlisting}[caption={Creating the plot of all characters.}, label={lstlisting:plot}] 
### Creating the final plot
plot(GOTgraph, edge.color=E(GOTgraph)$Type,# step 1: positive edges and negative edges get different colors
     vertex.color = V(GOTgraph)$Deadcolor, # step 2: dead characters become grey
     vertex.label.color = V(GOTgraph)$Deadcolor, # step 2
     edge.width = E(GOTgraph)$weightadj/3, # step 3: edges become thicker if two characters know each other better
     
     vertex.size = 1, #step 4: the nodes as such become small,...
     vertex.label.cex = log(GOTnodestrength)/5, # step 4 .... only the names become bigger based on their degree times the edge weights. The label size has been scaled by taking the natural logarithm of half the degree. This way, the small nodes don't diminish.
     vertex.label.dist = 1, # step 4, to have the labels next to the node
     
     mark.groups = GOTcluster, # step 5: mark the clusters
     mark.col = gray.colors(12, start = 0.6, alpha = 0.1), mark.border = NA, # step 5
     
     layout = GOTcoord, asp=0, #step 6,
     # with the coordinates by hand from tkplot() and without an aspect ratio
     
     ### Some layout adjustments
     vertex.frame.color = NA, # remove the frame from the vertex
     edge.curved = TRUE, # curved edges
     main = "A Game of Networks", # add a title
     sub = "Network of Game of Thrones Characters based on the first season of the series.
     Grey names depict dead characters by season 7.") 
# View this graph in the zoomed plot window and pull it to full size. The aspect ratio has been removed. 
\end{lstlisting}

\subsection{Global Measures}
The characters in \got seem to be fairly well connected. Not only does the world created by George R. R. Martin count 150 characters important enough to have shown up 3 or more times in the books or the show, it also has a large amount of edges and connections. Given the show expands over hundreds of miles which the characters walk by foot, it is remarkable that the average distance between two characters is $\langle d \rangle = 2.79$, see figure \ref{fig:plots1}. Thus when meeting a new person, every character on average knows someone that knows this new person. The people are well-connected even if it takes days to reach and visit each other. This is also displayed in the high density of 0.055.\footnote{Most networks are sparse with a much lower density \citep{RevModPhys.74.47}, a complete graph would have $L$ edges with $L=L_{max} = \binom{N}{2} = \tfrac{N(N-1)}{2} = 11,175$ in this case. This graph has 500 ties, even after the threshold of a minimum of 3 co-occurrences has been applied.}

\begin{figure}
    \centering
    \begin{subfigure}[c]{0.45\textwidth}
    \includegraphics[scale=0.4]{plots/degreedistribution.png}
    \end{subfigure}
    \begin{subfigure}[c]{0.45\textwidth}
    \includegraphics[scale=0.4]{plots/shortestpaths.png}
    \end{subfigure}
    \caption{Distributions of degrees and shortest paths of the network.}
    \label{fig:plots1}
\end{figure}

On average, each characters knows $\langle k \rangle = 8.187$ other characters, though most characters know 5 or less people, see figure \ref{fig:plots1}. The distribution of degrees $p_k$ follows a power law distribution,\footnote{$\alpha = 2.170$, the Kolmogorov-Smirnov test couldn't reject a power law distribution.} as it is common among sparse networks \citep{RevModPhys.74.47}. This means that the network includes a number of hubs, very central vertices that connect other nodes. Given that this is a classical fictional story that has to have some focus on a set of characters, the occurrence of hubs or main characters isn't too surprising. 

\begin{table}[h]
    \centering
    \begin{tabular}{lcr}
    \toprule
    \textbf{Measure} & & \\
    \midrule
    Order (Number of vertices) & $|V| = n$	&	150 \\
    Size (Number of edges) & $|E| = m$	&	614 \\
    Density for undirected networks & $\tfrac{2m}{n(n-1)}$	&	0,055\\
    Clustering coefficient/Transitivity & &	0,365\\
    Diameter of $G$ (length of longest shortest path) & &	48\\
    Average distance & $\langle d \rangle$ &	2,79\\
    Average degree & $\langle k \rangle$ & 8.187\\
    \bottomrule
    \end{tabular}
    \caption{Global measures of the network.}
    \label{tab:globalmeasures}
\end{table}

\newpage
\subsection{Local Measures}
Let's take a closer look at the characters itself and especially the most central and the best-connected people. Given the network in figure \ref{fig:GOTplot}, there are some very firm candidates for being a hub in \got. Most of the central characters are  from the Lannister and Stark family - like Tyrion Lannister, Eddard Stark, Sansa Stark or Robb Stark, all of them listed in high positions in table \ref{table:comparison} that compares the centrality of the characters. Family ties and a supportive network are a tremendous gift in times of trouble like present in \got, and the two big families own the support by birth.\\
On the other hand, Daenerys Targaryen and Jon Snow seem to be bridges for two communities outside of the main plot. The grey areas determine the subgroups found by \texttt{cluster\_walktrap()} which correspond quite well to the two communities Daenerys and Jon live in. When Jon joined the very remote \textit{Night's Watch} to protect the realm from anything north of the wall, he gained a network among the guards at the wall. Coming from the House of Stark, he nevertheless has strong connections to his family as well. He even ranks third among the strength of positive connections, see table \ref{table:comparison}. Daenerys, the former mad king's daughter, lives rather isolated with the folk of the \textit{Dothraki}. By being far away from King's Landing and the present king, she doesn't run the risk of being killed by the intrigues around the throne. Both Jon and Daenerys survived till season 7, surviving Robert Baratheon,\footnote{Robert had quite a positive network. It should be mentioned that Robert died while hunting and being drunk.} his elder son Joffrey Baratheon and his younger son Tommen Baratheon as kings, not to mention the numerous claims to the throne such as by Stannis Baratheon and Renley Baratheon. An explanation of the failed claims could be the lack of overwhelming support other characters rely on, as both are not well connected. In the most recent season, Jon and Daenerys met and fell in love while both of them have built their network further, gaining more and more support. The isolation of both in the first season seems to be a strong indicator for later success. A wide range of fans currently suspects the two to sit on the Iron Throne in the end \citep{ziss_9_2018}.\\
A tragic character of the very first season is Eddard Stark. According to degree, strength and eigencentrality, he is the most central character, ranking 2 in betweenness as well, see figure \ref{table:comparison}. When the old hand of the king\footnote{The advisor to the king and the most important person in the realm besides the king himself.} died, Robert Baratheon, the king, turned to his old friend Ned for help. Ned couldn't possibly neglect the offer, so the honorable Lord Stark made his way to King's Landing. Soon enough, his honesty endangered powerful intrigues to be disclosed, resulting in a devastating loss of support for him. Quite a few people wished to see Ned Stark dead, his \textit{StrengthMinus} value with only the negative relationships is higher than any other character's. Ned Stark seems to have gotten too deep into the network of Thrones while he righteously intended to reveal a number of dark secrets. This probably is the reason for him being killed after all in the middle of a very aversive network around him. Unfortunately for House Stark, the Lannisters make sure to not only kill Ned Stark but also his many supporters and guards, indicated by the many grey names among the Stark family in figure \ref{fig:GOTplot}.\\
His fate is in a way comparable to Joffrey Baratheon's, Petyr Baelish's and Tywin Lannister's fate. Though in contrast to Lord Stark, all of the three more or less openly met other characters with distrust, hate and aversion, they managed to build a very negative network as well. Hence, one of their opponents seems to have been witty or strong enough to murder them. This also means that social ties are connected: if the heir of House Stark, Sansa Stark in this case, is against Petyr Baelish, she will turn other people’s attitudes  towards him as well. All four characters impressively show the importance of a supportive network, especially when one ranks high in centrality of the network and becomes the focus of attention.\\
Varys, a character known for his connections similar to Petyr Baelish, mostly uses his influence from the background. His network expands among the unknown characters who deliver the most recent and important bits of information to him. It is that unimportant that he is not even present in table \ref{table:comparison}. In contrast to Petyr Baelish, he manages to mostly employ positive relationships with the important people and by that ensures his survival in \got.\\
Among the most central nodes, an important name to mention is Tyrion Lannister. Born into a family of intrigues, he uses his wits to survive the never-ending game of survival. During the course of the seasons, he doesn't only become hand of the king twice, he also keeps a dense network of supporters and allies. Given his dwarfism, his family isn't too fond of him, but tolerates him to a certain degree.\\
When drawn together, the living actors give an interesting picture: At the end of season 7, the main forces are the side of King's Landing with Cersei and her guards as well as Daenerys and Jon. Currently, Varys, Tyrion and quite a few other characters have sided with Daenerys. With Jon involved, it could even be possible to have the Stark family on their side. Jon on the other hand will inherit major parts of his father Ned Stark's network as he is the oldest surviving son. This leads to an overwhelmingly strong opponent to the queen Cersei. If there was an ending to the show with a king or queen involved, already the network of the very first season gives away a good estimate: Daenerys and Jon.\\
Being alive and becoming king are densely interconnected. Once a certain amount of power has emerged, one would have to make sure to survive by keeping up positive relationships. Then, one has high chances to sit on the iron throne.\\
Lastly, if the show actually had a very low number of main characters like other books or shows, the graph in figure \ref{fig:measurecomparison} might look less scattered as these nodes would then be clearly denoted as central. It shows how the different centrality measures don't agree on the same nodes. Eddard Stark in dark blue and the king Robert Baratheon in blue are overall very central, the \textit{StrengthRatio} however denotes Bran Stark, Jon Snow and Robb Stark with an impressively positive network as well. Bran Stark has emerged to a very important role in season 7 while Robb fought long and hard for becoming King in the North. Robb's downfall was also due to an intrigue, Bran is still alive. Clearly, the \textit{StrengthRatio} enforces the importance of taking positive and negative edges into consideration. Despite some actors like Eddard Stark being very central, other nodes have a much higher \textit{StrengthRatio} resulting in the peak in figure \ref{fig:measurecomparison}. This is how especially Eddard Stark might have been too central in the network which lead to him being killed.

\begin{figure}
    \centering
    \includegraphics[scale=0.34]{plots/measurecomparison.png}
    \caption{The 15 characters with the highest degree are ranked across different local measures that somewhat represent influence and connectedness. For a better overview, a logarithmic scale was chosen. \textit{StrengthPlus} and \textit{StrengthMinus} only take the positive or negative relationships into account, hence \textit{StrengthRatio} is the ratio of the two. As the bottom of the graph usually represents a better ranking, alive characters correspond to 1, dead characters to 2 on the \textit{Dead} scale.}
    \label{fig:measurecomparison}
\end{figure}


\begin{table}
\caption{Centrality comparison of the 15 characters with the highest degree, corresponding to figure \ref{fig:measurecomparison}. The table gives away the ranks of the characters in a certain centrality measure.}
    \label{table:comparison}
    \begin{adjustbox}{angle=270}
    \begin{tabular}
    {p{2.5cm}p{0.9cm}p{1.2cm}p{1.2cm}p{0.9cm}p{1.2cm}p{1.2cm}p{1.2cm}p{1.2cm}p{0.9cm}}
    \toprule
    Names & Degree & Strength & Between-ness & Close-ness & Eigen-centrality & Strength Plus & Strength Minus & Strength Ratio & Dead\\
    \midrule
    Eddard-Stark & 1 & 1 & 2 & 9 & 1 & 1 & 1 & 36 & 0\\
    Robert-Baratheon & 2 & 2 & 1 & 1 & 2 & 2 & 17 & 8 & 0\\
    Tyrion-Lannister & 3 & 4 & 5 & 6 & 10 & 8 & 4 & 50 & 1\\
    Catelyn-Stark & 4 & 8 & 11 & 40 & 6 & 7 & 7 & 38 & 0\\
    Jon-Snow & 5 & 3 & 8 & 32 & 8 & 3 & 15 & 14 & 1\\
    \addlinespace
    Robb-Stark & 6 & 7 & 4 & 17 & 13 & 5 & 18 & 17 & 0\\
    Sansa-Stark & 6 & 5 & 10 & 5 & 4 & 6 & 6 & 41 & 1\\
    Joffrey-Baratheon & 7 & 11 & 14 & 10 & 7 & 16 & 2 & 68 & 0\\
    Bran-Stark & 8 & 6 & 17 & 16 & 12 & 4 & 38 & 2 & 1\\
    Cersei-Lannister & 8 & 12 & 31 & 22 & 3 & 13 & 5 & 61 & 1\\
    \addlinespace
    Jaime-Lannister & 9 & 15 & 6 & 2 & 16 & 17 & 10 & 52 & 1\\
    Arya-Stark & 10 & 10 & 26 & 11 & 11 & 9 & 13 & 30 & 1\\
    Petyr-Baelish & 10 & 13 & 25 & 12 & 5 & 33 & 3 & 73 & 0\\
    Daenerys-Targaryen & 11 & 9 & 16 & 57 & 37 & 10 & 9 & 37 & 1\\
    Tywin-Lannister & 11 & 18 & 13 & 13 & 20 & 31 & 8 & 66 & 0\\
    \bottomrule
    \end{tabular}
    \end{adjustbox}
\end{table}
 % this inserts the big plot and the table